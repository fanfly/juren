\documentclass[10pt]{article}
\usepackage[paperwidth=105mm,paperheight=148mm,hmargin=1cm,top=1cm,bottom=1.75cm]{geometry}
\usepackage{fontspec}
\usepackage[PunctStyle=plain]{xeCJK}
\usepackage[shortlabels,inline]{enumitem}
\usepackage{amsmath}
\usepackage{amssymb}
\usepackage{caption}
\usepackage{chemformula}
\usepackage{siunitx}
\usepackage{tikz}
\usepackage{xeCJKfntef}
\usepackage{pgfplots}
\usepackage{lastpage}
\usepackage{fancyhdr}
\usepackage{needspace}

\pretolerance=5000
\tolerance=9000
\emergencystretch=10pt

\pagestyle{fancy}
\renewcommand{\headrulewidth}{0pt}
\fancyhead{}
\cfoot{\footnotesize\thepage\ /\ \pageref{LastPage}}
\setlength{\parindent}{0pt}
\setCJKmainfont[BoldFont=Noto Serif CJK TC SemiBold]{Noto Serif CJK TC}
\linespread{1.15}

\usepackage{tabto}
\newenvironment{choice}{\begin{enumerate}[label=(\Alph*),align=left,leftmargin=*,labelsep=.3em,topsep=0ex,itemsep=0ex]}{\end{enumerate}}
%\newenvironment{choices}[1]{\par\NumTabs{#1}\begin{enumerate*}[label=(\Alph*),itemjoin=\tab]}{\end{enumerate*}}

\newcommand*{\blank}[1]{\rule[-.7\baselineskip]{0pt}{1.8\baselineskip}\fbox{\rule[-.4\baselineskip]{0pt}{1.2\baselineskip}\hspace{#1}}}
\newcommand*{\fraction}[2]{\rule[-.8\baselineskip]{0pt}{2\baselineskip}\dfrac{#1}{#2}}
\newcommand*{\sfraction}[2]{\rule[-.4\baselineskip]{0pt}{1.2\baselineskip}\tfrac{#1}{#2}}

\usepackage{titlesec}
\usepackage{zhnumber}
\titleformat{\section}{\normalfont\bfseries}{{\thesection}、}{0em}{}
\titlespacing{\section}{0pt}{2ex plus .5ex minus .2ex}{1.3ex plus .2ex}
\renewcommand{\thesection}{\zhnum{section}}

\makeatletter
\renewcommand*{\maketitle}{{%
  \bfseries
  \LARGE 練習(數學) \\
  \large 數列、級數 \par
}}
\makeatother

\begin{document}
\maketitle
\medskip
答案位於第 \pageref{answer} 頁。
\section{選擇題(共 2 題)}
\begin{enumerate}[label=\arabic*.,align=left,leftmargin=*,labelsep=.3em]
  \item 設 $k \neq 1$,則 $1 + k + k^2 + \cdots + k^{113}$ 與下列何者相等?(單選)
  \begin{enumerate}[label=(\Alph*),align=left,leftmargin=*,labelsep=.3em]
    \item $k^{114}$
    \item $\fraction{k^{113}-1}{k-1}$
    \item $\fraction{k^{114}-1}{k-1}$
    \item $\fraction{k-1}{k^{113}-1}$
    \item $\fraction{k-1}{k^{114}-1}$
  \end{enumerate}
  \newpage
  \item 已知 $\langle a_n \rangle$ 為一實數數列。下列敘述何者正確?(多選)
  \begin{enumerate}[label=(\Alph*),align=left,leftmargin=*,labelsep=.3em]
    \item 若 $a_1 = \fraction{1}{2}$,$a_2 = \fraction{2}{3}$,$a_3 = \fraction{3}{4}$,則 $a_{2024} = \fraction{2024}{2025}$
    \item 若 $a_1, a_2, a_3$ 成等差,則 $\fraction{-a_1}{2}, \fraction{-a_2}{2}, \fraction{-a_3}{2}$ 成等差
    \item 若 $a_1, a_2, a_3$ 成等差,則 $2^{a_1}, 2^{a_2}, 2^{a_3}$ 成等比
    \item 若 $a_1, a_2, a_3$ 成等比,則 $\fraction{1}{a_1}, \fraction{1}{a_2}, \fraction{1}{a_3}$ 成等比
    \item 若 $a_1^2, a_2^2, a_3^2$ 成等比,則 $a_1, a_2, a_3$ 成等比
  \end{enumerate}
  \newpage
\end{enumerate}

\newpage
\label{answer}
{\bfseries\large 答案 \par}
\setcounter{section}{0}
\section{選擇題}
\begin{enumerate}[label=\arabic*.,left=0pt]
  \item (C)
  \item (B)(C)(D)
\end{enumerate}
\end{document}
