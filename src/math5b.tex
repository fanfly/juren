\documentclass[10pt]{article}
\usepackage[paperwidth=105mm,paperheight=148mm,hmargin=1cm,top=1cm,bottom=1.75cm]{geometry}
\usepackage{fontspec}
\usepackage[PunctStyle=plain]{xeCJK}
\usepackage[shortlabels,inline]{enumitem}
\usepackage{amsmath}
\usepackage{amssymb}
\usepackage{caption}
\usepackage{chemformula}
\usepackage{siunitx}
\usepackage{tikz}
\usepackage{xeCJKfntef}
\usepackage{pgfplots}
\usepackage{lastpage}
\usepackage{fancyhdr}
\usepackage{needspace}

\pretolerance=5000
\tolerance=9000
\emergencystretch=10pt

\pagestyle{fancy}
\renewcommand{\headrulewidth}{0pt}
\fancyhead{}
\cfoot{\footnotesize\thepage\ /\ \pageref{LastPage}}
\setlength{\parindent}{0pt}
\setCJKmainfont[BoldFont=Noto Serif CJK TC SemiBold]{Noto Serif CJK TC}
\linespread{1.15}

\usepackage{tabto}
\newenvironment{choice}{\begin{enumerate}[label=(\Alph*),align=left,leftmargin=*,labelsep=.3em,topsep=0ex,itemsep=0ex]}{\end{enumerate}}
%\newenvironment{choices}[1]{\par\NumTabs{#1}\begin{enumerate*}[label=(\Alph*),itemjoin=\tab]}{\end{enumerate*}}

\newcommand*{\blank}[1]{\rule[-.7\baselineskip]{0pt}{1.8\baselineskip}\fbox{\rule[-.4\baselineskip]{0pt}{1.2\baselineskip}\hspace{#1}}}
\newcommand*{\fraction}[2]{\rule[-.8\baselineskip]{0pt}{2\baselineskip}\dfrac{#1}{#2}}
\newcommand*{\sfraction}[2]{\rule[-.4\baselineskip]{0pt}{1.2\baselineskip}\tfrac{#1}{#2}}

\usepackage{titlesec}
\usepackage{zhnumber}
\titleformat{\section}{\normalfont\bfseries}{{\thesection}、}{0em}{}
\titlespacing{\section}{0pt}{2ex plus .5ex minus .2ex}{1.3ex plus .2ex}
\renewcommand{\thesection}{\zhnum{section}}

\makeatletter
\renewcommand*{\maketitle}{{%
  \bfseries
  \LARGE 練習(數學) \\
  \large 排列、組合 \par
}}
\makeatother

\begin{document}
\maketitle
\medskip
共有 3 題。答案位於第 \pageref{answer} 頁。
\section{選擇題}
\begin{enumerate}[label=\arabic*.,align=left,leftmargin=*,labelsep=.3em]
  \item 下列敘述何者正確?(多選)
  \begin{enumerate}[label=(\Alph*),align=left,leftmargin=*,labelsep=.3em]
    \item $C^6_0 + C^6_1 + C^6_2 + C^6_3 + C^6_4 + C^6_5 + C^6_6 = 64$
    \item $C^6_0 - C^6_1 + C^6_2 - C^6_3 + C^6_4 - C^6_5 + C^6_6 = 0$
    \item $C^6_0 - \fraction{C^6_1}{2} + \fraction{C^6_2}{2^2} - \fraction{C^6_3}{2^3} + \fraction{C^6_4}{2^4} - \fraction{C^6_5}{2^5} + \fraction{C^6_6}{2^6} = \fraction{-1}{64}$
    \item $(1-x)^6$ 的展開式中,$x^3$ 項的係數為 $20$
    \item $\biggl(x^2-\fraction{1}{2}\biggr)^{\mkern-4mu6}$ 的展開式中,$x^6$ 項的係數為 $\fraction{-5}{2}$
  \end{enumerate}
\end{enumerate}

\newpage

\section{填充題}
\begin{enumerate}[label=\arabic*.,align=left,leftmargin=*,labelsep=.3em,start=2]
  \item 甲、乙、丙、丁、戊、己、庚、辛、壬共 9 人想進行三對三的籃球比賽,因此打算要分成三隊,每隊各 3 人。
  \begin{enumerate}[label=(\arabic*),align=left,leftmargin=*,labelsep=.3em]
    \item 共有 $\blank{4em}$ 種分隊方法。
    \item 若甲、乙要在不同隊,則會有 $\blank{4em}$ 種分隊方法。
  \end{enumerate}
  \newpage
  \item 今要將 1、2、3、4、5、6、7、8、9 等 9 個數字排成(數字不重複的)九位數。
  \begin{enumerate}[label=(\arabic*),align=left,leftmargin=*,labelsep=.3em]
    \item 若要使數字是奇數的位數均不相鄰,則有 $\blank{4em}$ 個滿足條件的九位數。
    \item 若要使九位數當中的前五位數由左至右遞增,且後五位數由左至右遞減,則有 $\blank{4em}$ 個滿足條件的九位數。
  \end{enumerate}
\end{enumerate}

\newpage
\label{answer}
{\bfseries\large 答案 \par}
\begin{enumerate}[label=\arabic*.,left=0pt]
  \item (A)(B)(E)
  \item
  \begin{enumerate}[label=(\arabic*),align=left,leftmargin=*,labelsep=.3em]
    \item 280
    \item 210
  \end{enumerate}
  \item
  \begin{enumerate}[label=(\arabic*),align=left,leftmargin=*,labelsep=.3em]
    \item 2880
    \item 70
  \end{enumerate}
\end{enumerate}
\end{document}
