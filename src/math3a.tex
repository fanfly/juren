\documentclass[10pt]{article}
\usepackage[paperwidth=105mm,paperheight=148mm,hmargin=1cm,top=1cm,bottom=1.75cm]{geometry}
\usepackage{fontspec}
\usepackage[PunctStyle=plain]{xeCJK}
\usepackage[shortlabels,inline]{enumitem}
\usepackage{amsmath}
\usepackage{amssymb}
\usepackage{caption}
\usepackage{chemformula}
\usepackage{siunitx}
\usepackage{tikz}
\usepackage{xeCJKfntef}
\usepackage{pgfplots}
\usepackage{lastpage}
\usepackage{fancyhdr}
\usepackage{needspace}

\pretolerance=5000
\tolerance=9000
\emergencystretch=10pt

\pagestyle{fancy}
\renewcommand{\headrulewidth}{0pt}
\fancyhead{}
\cfoot{\footnotesize\thepage\ /\ \pageref{LastPage}}
\setlength{\parindent}{0pt}
\setCJKmainfont[BoldFont=Noto Serif CJK TC SemiBold]{Noto Serif CJK TC}
\linespread{1.15}

\usepackage{tabto}
\newenvironment{choice}{\begin{enumerate}[label=(\Alph*),align=left,leftmargin=*,labelsep=.3em,topsep=0ex,itemsep=0ex]}{\end{enumerate}}
%\newenvironment{choices}[1]{\par\NumTabs{#1}\begin{enumerate*}[label=(\Alph*),itemjoin=\tab]}{\end{enumerate*}}

\newcommand*{\blank}[1]{\rule[-.7\baselineskip]{0pt}{1.8\baselineskip}\fbox{\rule[-.4\baselineskip]{0pt}{1.2\baselineskip}\hspace{#1}}}
\newcommand*{\fraction}[2]{\rule[-.8\baselineskip]{0pt}{2\baselineskip}\dfrac{#1}{#2}}
\newcommand*{\sfraction}[2]{\rule[-.4\baselineskip]{0pt}{1.2\baselineskip}\tfrac{#1}{#2}}

\usepackage{titlesec}
\usepackage{zhnumber}
\titleformat{\section}{\normalfont\bfseries}{{\thesection}、}{0em}{}
\titlespacing{\section}{0pt}{2ex plus .5ex minus .2ex}{1.3ex plus .2ex}
\renewcommand{\thesection}{\zhnum{section}}

\makeatletter
\renewcommand*{\maketitle}{{%
  \bfseries
  \LARGE 練習(數學) \\
  \large 多項式的除法、一次與二次函數 \par
}}
\makeatother

\begin{document}
\maketitle
\medskip
第 \pageref{hint} 頁設有提示,答案位於第 \pageref{answer} 頁。
\section{填充題(共 3 題)}
\begin{enumerate}[label=\Alph*.,align=left,leftmargin=*,labelsep=.3em]
  \item 設 $f(x)$ 為二次多項式,其滿足 $f(0) = 1$,$f(1) = 3$,$f(2) = 9$。則 $f(3) = \blank{5em}$。
  \newpage
  \item 已知 $f(x) = x^4 - 20x^3 + 158x^2 - 580x + 825$,則 $f(5 + \sqrt{2}) = \blank{5em}$。
  \newpage
  \item 設 $f(x)$ 為一多項式,且 $x^2f(x)$ 除以 $x^2 + 2$ 的餘式為 $-20x - 24$。則 $f(x)$ 除以 $x^2 + 2$ 的餘式為 $\blank{8em}$。
\end{enumerate}

\newpage
\label{hint}
{\bfseries\large 提示 \par}
\setcounter{section}{0}
\section{填充題}
\begin{enumerate}[label=\Alph*.,left=0pt]
  \item 由於 $f(x)$ 是二次多項式,我們可以假設 $f(x)$ 形如 $ax^2 + bx + c$。
  \item 我們可以利用綜合除法把 $f(x)$ 轉換為 $a(x-5)^4 + b(x-5)^3 + c(x-5)^2 + d(x-5) + 5$ 的形式。
  \item 我們可以假設 $f(x)$ 除以 $x^2 + 2$ 的餘式是 $ax + b$,並觀察此時 $x^2f(x)$ 除以 $x^2 + 2$ 的餘式如何用 $a$、$b$ 表示。
\end{enumerate}

\newpage
\label{answer}
{\bfseries\large 答案 \par}
\setcounter{section}{0}
\section{填充題}
\begin{enumerate}[label=\Alph*.,left=0pt]
  \item 19
  \item 20
  \item $10x + 12$
\end{enumerate}
\end{document}
