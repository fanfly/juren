\documentclass[10pt]{article}
\usepackage[paperwidth=105mm,paperheight=148mm,hmargin=1cm,top=1cm,bottom=1.75cm]{geometry}
\usepackage{fontspec}
\usepackage[PunctStyle=plain]{xeCJK}
\usepackage[shortlabels,inline]{enumitem}
\usepackage{amsmath}
\usepackage{amssymb}
\usepackage{caption}
\usepackage{chemformula}
\usepackage{siunitx}
\usepackage{tikz}
\usepackage{xeCJKfntef}
\usepackage{pgfplots}
\usepackage{lastpage}
\usepackage{fancyhdr}
\usepackage{needspace}

\pretolerance=5000
\tolerance=9000
\emergencystretch=10pt

\pagestyle{fancy}
\renewcommand{\headrulewidth}{0pt}
\fancyhead{}
\cfoot{\footnotesize\thepage\ /\ \pageref{LastPage}}
\setlength{\parindent}{0pt}
\setCJKmainfont[BoldFont=Noto Serif CJK TC SemiBold]{Noto Serif CJK TC}
\linespread{1.15}

\usepackage{tabto}
\newenvironment{choice}{\begin{enumerate}[label=(\Alph*),align=left,leftmargin=*,labelsep=.3em,topsep=0ex,itemsep=0ex]}{\end{enumerate}}
%\newenvironment{choices}[1]{\par\NumTabs{#1}\begin{enumerate*}[label=(\Alph*),itemjoin=\tab]}{\end{enumerate*}}

\newcommand*{\blank}[1]{\rule[-.7\baselineskip]{0pt}{1.8\baselineskip}\fbox{\rule[-.4\baselineskip]{0pt}{1.2\baselineskip}\hspace{#1}}}
\newcommand*{\fraction}[2]{\rule[-.8\baselineskip]{0pt}{2\baselineskip}\dfrac{#1}{#2}}
\newcommand*{\sfraction}[2]{\rule[-.4\baselineskip]{0pt}{1.2\baselineskip}\tfrac{#1}{#2}}

\usepackage{titlesec}
\usepackage{zhnumber}
\titleformat{\section}{\normalfont\bfseries}{{\thesection}、}{0em}{}
\titlespacing{\section}{0pt}{2ex plus .5ex minus .2ex}{1.3ex plus .2ex}
\renewcommand{\thesection}{\zhnum{section}}

\makeatletter
\renewcommand*{\maketitle}{{%
  \bfseries
  \LARGE 練習(數學) \\
  \large 直角三角形的三角比 \par
}}
\makeatother

\begin{document}
\maketitle
\medskip
第 \pageref{hint} 頁設有提示,答案位於第 \pageref{answer} 頁。
\section{填充題(共 3 題)}
\begin{enumerate}[label=\Alph*.,align=left,leftmargin=*,labelsep=.3em]
  \item 已知 $0^\circ < \alpha < 90^\circ$,且 $\sin\alpha = \fraction{1}{7}$。則:
  \begin{enumerate}[label=(\arabic*),left=0pt]
    \item $\cos\alpha = \blank{5em}$。(化為最簡根式)
    \item $\tan\alpha = \blank{5em}$。(化為最簡根式)
  \end{enumerate}
  \newpage
  \item 已知 $0^\circ < \theta < 90^\circ$,且 $\sin\theta + \cos\theta = \fraction{23}{17}$。則:
  \begin{enumerate}[label=(\arabic*),left=0pt]
    \item $\sin\theta \cdot \cos\theta = \blank{5em}$。
    \item $\sin\theta - \cos\theta = \blank{5em}$。
    \item 若 $\theta < 45^\circ$,則 $\sin\theta - \cos\theta = \blank{5em}$。
  \end{enumerate}
  \newpage
  \item 試求 $\sin^2 40^\circ + \sin^2 50^\circ + \sin^2 60^\circ = \blank{5em}$。
\end{enumerate}

\newpage
\label{hint}
{\bfseries\large 提示 \par}
\setcounter{section}{0}
\section{填充題}
\begin{enumerate}[label=\Alph*.,left=0pt]
  \item
  \begin{enumerate}[label=(\arabic*),left=0pt]
    \item 利用平方關係:$\sin^2\alpha + \cos^2\alpha = 1$。
    \item 利用商數關係:$\tan \alpha = \fraction{\sin\alpha}{\cos\alpha}$。
  \end{enumerate}
  \item
  \begin{enumerate}[label=(\arabic*),left=0pt]
    \item 觀察 $(\sin \theta + \cos \theta)^2$ 的展開式,並利用平方關係 $\sin^2\theta + \cos^2\theta = 1$。
    \item 觀察 $(\sin \theta - \cos \theta)^2$ 的展開式,並利用平方關係 $\sin^2\theta + \cos^2\theta = 1$。
    \item $\theta < 45^\circ$ 時,$\sin\theta < \fraction{1}{\sqrt{2}}$,且 $\cos\theta > \fraction{1}{\sqrt{2}}$。
  \end{enumerate}
  \item 先計算 $\sin^240^\circ+\sin^250^\circ$,利用餘角關係與平方關係可以求得其值。接著計算 $\sin^260^\circ$ 即可。
\end{enumerate}

\newpage
\label{answer}
{\bfseries\large 答案 \par}
\setcounter{section}{0}
\section{填充題}
\begin{enumerate}[label=\Alph*.,left=0pt]
  \item
  \begin{enumerate}[label=(\arabic*),left=0pt]
    \item $\fraction{4\sqrt{3}}{7}$
    \item $\fraction{\sqrt{3}}{12}$
  \end{enumerate}
  \item
  \begin{enumerate}[label=(\arabic*),left=0pt]
    \item $\fraction{120}{289}$
    \item $\pm\fraction{7}{17}$
    \item $-\fraction{7}{17}$
  \end{enumerate}
  \item $\fraction{7}{4}$
\end{enumerate}
\end{document}
