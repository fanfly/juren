\documentclass[12pt]{article}
\usepackage[paperwidth=148mm,paperheight=210mm,hmargin=1cm,top=1cm,bottom=1.75cm]{geometry}
\usepackage{fontspec}
\usepackage{xeCJK}
\usepackage[shortlabels,inline]{enumitem}
\usepackage{amsmath}
\usepackage{amssymb}
\usepackage{caption}
\usepackage{chemformula}
\usepackage{siunitx}
\usepackage{tikz}
\usepackage{xeCJKfntef}
\usepackage{pgfplots}
\usepackage{lastpage}
\usepackage{fancyhdr}
\usepackage{needspace}

\pretolerance=5000
\tolerance=9000
\emergencystretch=10pt

\pagestyle{fancy}
\renewcommand{\headrulewidth}{0pt}
\fancyhead{}
\cfoot{\footnotesize\thepage\ /\ \pageref{LastPage}}
\setlength{\parindent}{0pt}
\setCJKmainfont[BoldFont=Noto Serif CJK TC SemiBold]{Noto Serif CJK TC}
%\setCJKmainfont{AR PL UKai TW}
\linespread{1.15}

\usepackage{tabto}
\newenvironment{choices}[1]{\par\NumTabs{#1}\begin{enumerate*}[label=(\Alph*),itemjoin=\tab]}{\end{enumerate*}}

\newcommand*{\blank}[1]{\rule[-.7\baselineskip]{0pt}{1.8\baselineskip}\fbox{\rule[-.4\baselineskip]{0pt}{1.2\baselineskip}\hspace{#1}}}
\newcommand*{\fraction}[2]{\rule[-.8\baselineskip]{0pt}{2\baselineskip}\dfrac{#1}{#2}}

\usepackage{titlesec}
\usepackage{zhnumber}
\titleformat{\section}{\normalfont\bfseries}{\thesection{、}}{0em}{}
\titlespacing{\section}{0pt}{2ex plus .5ex minus .2ex}{1.3ex plus .2ex}
\renewcommand{\thesection}{\zhnum{section}}

%\title{高一上第一次期中考數學科模擬試題}
\makeatletter
\renewcommand*{\maketitle}{{%
  \bfseries
  \LARGE 練習題 \\
  \large 範圍:實數、絕對值、指數、常用對數 \par
}}
\makeatother

\begin{document}
\maketitle
\section{單選題}
\begin{enumerate}[align=left,leftmargin=*,labelsep=.6em]
  \needspace{3\baselineskip}
  \item 將 $\fraction{5}{37}$ 化為小數後,其小數點後第 2023 位數字為何?
  \begin{choices}{3}
    \item 1
    \item 2
    \item 3
    \item 4
    \item 5
  \end{choices}
  \needspace{5\baselineskip}
  \item 已知 $a = \fraction{2\sqrt{11}+\sqrt{13}}{3}$,$b = \fraction{3\sqrt{11}+2\sqrt{13}}{5}$,$c = \fraction{4\sqrt{11}+3\sqrt{13}}{7}$,則 $a$、$b$、$c$ 的大小關係為何?
  \begin{choices}{3}
    \item $a < b < c$
    \item $a < c < b$
    \item $b < a < c$
    \item $c < a < b$
    \item $c < b < a$
  \end{choices}
  \needspace{3\baselineskip}
  \item 若 $\log(\log a) = -1$,則 $(0.1)^{0.1}$ 與下列何者相等?
  \begin{choices}{3}
    \item $a$
    \item $\fraction{1}{a}$
    \item $\fraction{a}{10}$
    \item $\fraction{10}{a}$
    \item $\fraction{1}{10a}$
  \end{choices}
\end{enumerate}

\needspace{8\baselineskip}
\section{多選題}
\begin{enumerate}[align=left,leftmargin=*,labelsep=.6em,resume]
  \item 下列敘述何者正確?
  \begin{choices}{1}
    \item 若 $x$ 為有理數,則 $x^2$ 為有理數
    \item 若 $x$ 為無理數,則 $x^2$ 為無理數
    \item 若 $x^2$ 與 $x^3$ 皆為有理數,則 $x$ 為有理數
    \item 若 $x+y$ 與 $x-y$ 皆為有理數,則 $x$ 與 $y$ 皆為有理數
    \item 若 $x+y$ 與 $xy$ 皆為有理數,則 $x$ 與 $y$ 皆為有理數
  \end{choices}
  \needspace{6\baselineskip}
  \item 若 $x$、$y$ 兩實數滿足 $-1 \leq x \leq 3$ 與 $2 \leq y \leq 4$,則下列敘述何者正確?
  \begin{choices}{1}
    \item $1 \leq x + y \leq 7$
    \item $-5 \leq x - y \leq 1$
    \item $-2 \leq xy \leq 12$
    \item $3 \leq \lvert x \rvert + \lvert y \rvert \leq 7$
    \item $4 \leq x^2 + y^2 \leq 25$
  \end{choices}
  \needspace{6\baselineskip}
  \item 已知 $0.3010 < \log 2 < 0.3011$。下列敘述何者正確?
  \begin{choices}{1}
    \item $10^{\log 2} > 1$
    \item $10^{0.3} < 2$
    \item $10^{0.7} < 5$
    \item $2^{100}$ 為 30 位數
    \item $2^{-100}$ 在小數點後第 30 位開始不為 0
  \end{choices}
\end{enumerate}

\needspace{10\baselineskip}
\section{填充題}
\begin{enumerate}[label=\Alph*.,align=left,leftmargin=*,labelsep=.6em]
  \needspace{3\baselineskip}
  \item $(\sqrt[3]{25}+\sqrt[3]{5}+1)(\sqrt[3]{5}-1) = \blank{8em}$。
  \needspace{3\baselineskip}
  \item 數線上有 $A(-5)$、$B(11)$、$C(c)$ 三點,若 $\overline{AC}:\overline{BC} = 3:5$,則 $c = \blank{8em}$。
  \needspace{3\baselineskip}
  \item 不等式 $\lvert 2x + 3 \rvert \leq \lvert x - 1 \rvert$ 之解為 \blank{10em}。
  \needspace{3\baselineskip}
  \item 若不等式 $\lvert ax + 4 \rvert \geq b$ 的解為 $x \leq -1$ 或 $x \geq 9$,則數對 $(a, b)$ 為 \blank{8em}。
  \needspace{3\baselineskip}
  \item 若 $0 < a < 1$ 且 $a + a^{-1} = 7$,則 $a^3 + a^{-3} = \blank{8em}$,$a - a^{-1} = \blank{8em}$。
  \needspace{3\baselineskip}
  \item 設 $\sqrt{11+4\sqrt{7}}$ 的整數部分為 $a$,小數部分為 $b$,則 $\fraction{1}{b} - \fraction{1}{a+b} = \blank{8em}$。
  \needspace{3\baselineskip}
  \item 已知 $x$、$y$ 兩實數滿足 $7^x = 27$ 與 $63^y = 81$,則 $\fraction{3}{x} - \fraction{4}{y} = \blank{8em}$。
  \needspace{3\baselineskip}
  \item 設 $y = 5^x + 5^{1-x}$,其中 $x$ 為實數。若在 $x = a$ 時,$y$ 具有最小值 $b$,則數對 $(a, b)$ 為 \blank{8em}。
  \needspace{3\baselineskip}
  \item 若一放射性物質的半衰期為 $T$,則每經過時間 $T$,該物質的質量會衰退為原來的一半。已知在一實驗開始時,A、B 兩種放射性物質的質量相同,且經過 72 小時後,物質 A、B 的質量比變為 $1:4$。若物質 A 的半衰期為 12 小時,則物質 B 的半衰期為 \blank{8em} 小時。
\end{enumerate}

\needspace{5\baselineskip}
\section{混合題}
\begin{enumerate}[align=left,leftmargin=*,labelsep=.6em,parsep=0ex]
  \item 已知 $\sqrt{2}$ 為無理數。
  \begin{enumerate}[label=(\arabic*),align=left,leftmargin=*,labelsep=.4em]
    \item 試證明 $2\sqrt{2}$ 為無理數。
    \item 試證明 $\sqrt[4]{2}$ 為無理數。
  \end{enumerate}
  \needspace{6\baselineskip}
  \item 一實驗需要在培養皿中培養細菌,原先培養皿中共有 $6 \times 10^4$ 個細菌。已知經過 $k$ 日後,細菌數會變為原先的 $c^k$ 倍。
  \begin{enumerate}[label=(\arabic*),align=left,leftmargin=*,labelsep=.4em]
    \item 若經過 8 日後,培養皿中共有 $9.6 \times 10^5$ 個細菌,則 $c$ 之值為何?
    \item 承 (1),從開始培養起算,經過幾日後培養皿中的細菌數會達到 $3.84 \times 10^6$?
  \end{enumerate}
\end{enumerate}

\newpage
{\bfseries\large 答案 \par}
\section*{選擇題}
\NumTabs{3}
\begin{enumerate*}[label=\arabic*.,itemjoin=\tab]
  \item (A)
  \item (A)
  \item (B)
  \item (A)(C)(D)
  \item (A)(B)(E)
  \item (A)(B)
\end{enumerate*}
\section*{填充題}
\NumTabs{3}
\begin{enumerate*}[label=\Alph*.,itemjoin=\tab]
  \item 4
  \item $-29$ 或 $1$
  \item $-4 \leq x \leq -\fraction{2}{3}$
  \item $(-1,5)$
  \item $322$;$-3\sqrt{5}$
  \item $\fraction{4}{3}$
  \item $\fraction{1}{9}$
  \item $\biggl(\fraction{1}{2}, 2\sqrt{5}\biggr)$
  \item 18
\end{enumerate*}
\section*{混合題}
\begin{enumerate}[align=left,leftmargin=*,labelsep=.6em,parsep=0ex]
  \item
  \begin{enumerate}[label=(\arabic*),align=left,leftmargin=*,labelsep=.4em]
    \item 若 $2\sqrt{2}$ 為有理數,則 $2\sqrt{2} \times \fraction{1}{2} = \sqrt{2}$ 為有理數,矛盾。故 $2\sqrt{2}$ 為無理數。
    \item 若 $\sqrt[4]{2}$ 為有理數,則 $\sqrt[4]{2} \times \sqrt[4]{2} = \sqrt{2}$ 為有理數,矛盾。故 $\sqrt[4]{2}$ 為無理數。
  \end{enumerate}
  \item
  \begin{enumerate}[label=(\arabic*),align=left,leftmargin=*,labelsep=.4em]
    \item $\sqrt{2}$
    \item 12
  \end{enumerate}
\end{enumerate}
\end{document}
