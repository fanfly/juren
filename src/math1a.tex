\documentclass[12pt]{article}
\usepackage[paperwidth=148mm,paperheight=210mm,hmargin=1cm,top=1cm,bottom=1.75cm]{geometry}
\usepackage{fontspec}
\usepackage{xeCJK}
\usepackage[shortlabels,inline]{enumitem}
\usepackage{amsmath}
\usepackage{amssymb}
\usepackage{caption}
\usepackage{chemformula}
\usepackage{siunitx}
\usepackage{tikz}
\usepackage{xeCJKfntef}
\usepackage{pgfplots}
\usepackage{lastpage}
\usepackage{fancyhdr}
\usepackage{needspace}

\pretolerance=5000
\tolerance=9000
\emergencystretch=10pt

\pagestyle{fancy}
\renewcommand{\headrulewidth}{0pt}
\fancyhead{}
\cfoot{\footnotesize\thepage\ /\ \pageref{LastPage}}
\setlength{\parindent}{0pt}
\setCJKmainfont[BoldFont=Noto Serif CJK TC SemiBold]{Noto Serif CJK TC}
%\setCJKmainfont{AR PL UKai TW}
\linespread{1.15}

\usepackage{tabto}
\newenvironment{choices}[1]{\par\NumTabs{#1}\begin{enumerate*}[label=(\Alph*),itemjoin=\tab]}{\end{enumerate*}}

\newcommand*{\blank}[1]{\rule[-.7\baselineskip]{0pt}{1.8\baselineskip}\fbox{\rule[-.4\baselineskip]{0pt}{1.2\baselineskip}\hspace{#1}}}
\newcommand*{\fraction}[2]{\rule[-.7\baselineskip]{0pt}{1.8\baselineskip}\dfrac{#1}{#2}}

\usepackage{titlesec}
\usepackage{zhnumber}
\titleformat{\section}{\needspace{5\baselineskip}\normalfont\bfseries}{\thesection{、}}{0em}{}
\renewcommand{\thesection}{\zhnum{section}}

\title{高一上第一次期中考數學科模擬試題}
\makeatletter
\renewcommand*{\maketitle}{{%
  \bfseries
  \LARGE 數學科段考複習試題 \\
  \large 高一上第一次期中考範圍 \par
}}
\makeatother

\begin{document}
\maketitle
\section{單一選擇題(10\%)}
\begin{enumerate}[align=left,leftmargin=*,labelsep=.6em]
  \item 將 $\fraction{5}{37}$ 化為小數後,其小數點後第 2023 位數字為何?
  \begin{choices}{3}
    \item 1
    \item 2
    \item 3
    \item 4
    \item 5
  \end{choices}
  \item 已知 $a = \fraction{2\sqrt{11}+\sqrt{13}}{3}$,$b = \fraction{3\sqrt{11}+2\sqrt{13}}{5}$,$c = \fraction{4\sqrt{11}+3\sqrt{13}}{7}$,則 $a$、$b$、$c$ 的大小關係為何?
  \begin{choices}{3}
    \item $a < b < c$
    \item $a < c < b$
    \item $b < a < c$
    \item $c < a < b$
    \item $c < b < a$
  \end{choices}
\end{enumerate}

\section{多重選擇題(20\%)}
\begin{enumerate}[align=left,leftmargin=*,labelsep=.6em,resume]
  \item 若 $x$、$y$ 兩實數滿足 $-1 \leq x \leq 3$ 與 $2 \leq y \leq 4$,則下列敘述何者正確?
  \begin{choices}{2}
    \item $1 \leq x + y \leq 7$
    \item $-5 \leq x - y \leq 1$
    \item $-2 \leq xy \leq 12$
    \item $3 \leq \lvert x \rvert + \lvert y \rvert \leq 7$
    \item $2 \leq \sqrt{x^2 + y^2} \leq 5$
  \end{choices}
\end{enumerate}

\section{填充題(40\%)}
\begin{enumerate}[label=\Alph*.,align=left,leftmargin=*,labelsep=.6em]
  \item $(\sqrt[3]{25}+\sqrt[3]{5}+1)(\sqrt[3]{5}-1) = \blank{8em}$。
  \item 數線上有 $A(-5)$、$B(11)$、$C(c)$ 三點,若 $\overline{AC}:\overline{BC} = 3:5$,則 $c = \blank{8em}$。
  \item 設 $\sqrt{11+4\sqrt{7}}$ 的整數部分為 $a$,小數部分為 $b$,則 $\fraction{1}{b} - \fraction{1}{a+b} = \blank{8em}$。
  \item 已知 $x$、$y$ 兩實數滿足 $67^x = 27$ 與 $603^y = 81$,則 $\fraction{3}{x} - \fraction{4}{y} = \blank{8em}$。
  \item 設 $y = 5^x + 5^{1-x}$,其中 $x$ 為實數。若在 $x = a$ 時,$y$ 具有最小值 $b$,則數對 $(a, b)$ 為 \blank{8em}。
\end{enumerate}
\end{document}
