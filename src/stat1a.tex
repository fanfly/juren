\documentclass[10pt]{article}
\usepackage[paperwidth=105mm,paperheight=148mm,hmargin=1cm,top=1cm,bottom=1.75cm]{geometry}
\usepackage{fontspec}
\usepackage[PunctStyle=plain]{xeCJK}
\usepackage[shortlabels,inline]{enumitem}
\usepackage{amsmath}
\usepackage{amssymb}
\usepackage{caption}
\usepackage{chemformula}
\usepackage{siunitx}
\usepackage{tikz}
\usepackage{xeCJKfntef}
\usepackage{pgfplots}
\usepackage{lastpage}
\usepackage{fancyhdr}
\usepackage{needspace}

%\pretolerance=5000
%\tolerance=9000
\emergencystretch=10pt

\pagestyle{fancy}
\renewcommand{\headrulewidth}{0pt}
\fancyhead{}
\cfoot{\footnotesize\thepage\ /\ \pageref{LastPage}}
\setlength{\parindent}{0pt}
\setCJKmainfont[BoldFont=Noto Serif CJK TC SemiBold]{Noto Serif CJK TC}
\linespread{1.15}

\usepackage{tabto}
\newenvironment{choice}{\begin{enumerate}[label=(\Alph*),align=left,leftmargin=*,labelsep=.3em,topsep=0ex,itemsep=0ex]}{\end{enumerate}}
%\newenvironment{choices}[1]{\par\NumTabs{#1}\begin{enumerate*}[label=(\Alph*),itemjoin=\tab]}{\end{enumerate*}}

\newcommand*{\blank}[1]{\rule[-.7\baselineskip]{0pt}{1.8\baselineskip}\fbox{\rule[-.4\baselineskip]{0pt}{1.2\baselineskip}\hspace{#1}}}
\newcommand*{\fraction}[2]{\rule[-.8\baselineskip]{0pt}{2\baselineskip}\dfrac{#1}{#2}}
\newcommand*{\sfraction}[2]{\rule[-.4\baselineskip]{0pt}{1.2\baselineskip}\tfrac{#1}{#2}}

\usepackage{titlesec}
\usepackage{zhnumber}
\titleformat{\section}{\normalfont\bfseries}{{\thesection}、}{0em}{}
\titlespacing{\section}{0pt}{2ex plus .5ex minus .2ex}{1.3ex plus .2ex}
\renewcommand{\thesection}{\zhnum{section}}

\makeatletter
\renewcommand*{\maketitle}{{%
  \bfseries
  \LARGE 練習(統計) \\
  \large 離散型隨機變數、連續型隨機變數 \par
}}
\makeatother

\begin{document}
\maketitle
\medskip
答案位於第 \pageref{answer} 頁。
\begin{enumerate}[label=\arabic*.,align=left,leftmargin=*,labelsep=.5em]
  \item Assume that binomial distribution applies, the sample size is 4, and the probability of success is 0.8.
  Find the following probabilities.
  \begin{enumerate}[label=(\alph*),left=0pt,widest=a,topsep=0ex]
    \item The probability of exactly 2 successes.
    \item The probability of 3 or more successes.
  \end{enumerate}
  \newpage
  \item Assume that binomial distribution applies, the sample size is 150, and the probability of success is 0.6.
  \begin{enumerate}[label=(\alph*),left=0pt,widest=a,topsep=0ex]
    \item Find the expected value of the number of successes.
    \item Find the standard deviation of the number of successes.
  \end{enumerate}
  \newpage
  \item Customers arrive at a checkout counter in a department store according to a Poisson distribution
  at an average of 1 per ten minutes.
  During a given ten minutes, find the following probabilities.
  \begin{enumerate}[label=(\alph*),left=0pt,widest=a,topsep=0ex]
    \item The probability that exactly 1 customers arrive.
    \item The probability that at least 2 customers arrive.
  \end{enumerate}
\end{enumerate}

\newpage
\label{answer}
{\bfseries\large 答案 \par}
\NumTabs{4}
\begin{enumerate}[label=\arabic*.,align=left,leftmargin=*,labelsep=.3em]
  \item
  \begin{enumerate*}[label=(\alph*),itemjoin=\tab]
    \item 0.1536
    \item 0.8192
  \end{enumerate*}
  \item
  \begin{enumerate*}[label=(\alph*),itemjoin=\tab]
    \item 25
    \item 6
  \end{enumerate*}
  \item
  \begin{enumerate*}[label=(\alph*),itemjoin=\tab]
    \item 0.3679
    \item 0.2642
  \end{enumerate*}
\end{enumerate}
\end{document}
