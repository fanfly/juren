\documentclass[10pt]{article}
\usepackage[paperwidth=105mm,paperheight=148mm,hmargin=1cm,top=1cm,bottom=1.75cm]{geometry}
\usepackage{fontspec}
\usepackage[PunctStyle=plain]{xeCJK}
\usepackage[shortlabels,inline]{enumitem}
\usepackage{amsmath}
\usepackage{amssymb}
\usepackage{caption}
\usepackage{chemformula}
\usepackage{siunitx}
\usepackage{tikz}
\usepackage{xeCJKfntef}
\usepackage{pgfplots}
\usepackage{lastpage}
\usepackage{fancyhdr}
\usepackage{needspace}

\pretolerance=5000
\tolerance=9000
\emergencystretch=10pt

\pagestyle{fancy}
\renewcommand{\headrulewidth}{0pt}
\fancyhead{}
\cfoot{\footnotesize\thepage\ /\ \pageref{LastPage}}
\setlength{\parindent}{0pt}
\setCJKmainfont[BoldFont=Noto Serif CJK TC SemiBold]{Noto Serif CJK TC}
\linespread{1.15}

\usepackage{tabto}
\newenvironment{choice}{\begin{enumerate}[label=(\Alph*),align=left,leftmargin=*,labelsep=.3em,topsep=0ex,itemsep=0ex]}{\end{enumerate}}
%\newenvironment{choices}[1]{\par\NumTabs{#1}\begin{enumerate*}[label=(\Alph*),itemjoin=\tab]}{\end{enumerate*}}

\newcommand*{\blank}[1]{\rule[-.7\baselineskip]{0pt}{1.8\baselineskip}\fbox{\rule[-.4\baselineskip]{0pt}{1.2\baselineskip}\hspace{#1}}}
\newcommand*{\fraction}[2]{\rule[-.8\baselineskip]{0pt}{2\baselineskip}\dfrac{#1}{#2}}
\newcommand*{\sfraction}[2]{\rule[-.4\baselineskip]{0pt}{1.2\baselineskip}\tfrac{#1}{#2}}

\usepackage{titlesec}
\usepackage{zhnumber}
\titleformat{\section}{\normalfont\bfseries}{{\thesection}、}{0em}{}
\titlespacing{\section}{0pt}{2ex plus .5ex minus .2ex}{1.3ex plus .2ex}
\renewcommand{\thesection}{\zhnum{section}}

\makeatletter
\renewcommand*{\maketitle}{{%
  \bfseries
  \LARGE 練習(數學) \\
  \large 計數原理、排列 \par
}}
\makeatother

\begin{document}
\maketitle
\medskip
共有 3 題。答案位於第 \pageref{answer} 頁。
\section{填充題}
\begin{enumerate}[label=\arabic*.,align=left,leftmargin=*,labelsep=.3em]
  \item 桌上有編號 1、2、3、6 的球,我們要將四顆球全部放入甲、乙兩個箱子當中。
  \begin{enumerate}[label=(\arabic*),align=left,leftmargin=*,labelsep=.3em]
    \item 共有 $\blank{4em}$ 種放置方法。
    \item 若要使兩箱中都至少有 1 顆球,會有 $\blank{4em}$ 種放置方法。
    \item 若要使兩箱中都至少有 1 顆球,且甲、乙兩箱中球編號的乘積相等,會有 $\blank{4em}$ 種放置方法。
  \end{enumerate}
  \newpage
  \item 甲、乙、丙、丁、戊、己共 6 人打算要排成一列拍攝團體照。
  \begin{enumerate}[label=(\arabic*),align=left,leftmargin=*,labelsep=.3em]
    \item 共有 $\blank{5em}$ 種排法。
    \item 若甲、乙要相鄰,且丙、丁也要相鄰,則有 $\blank{5em}$ 種排法。
    \item 若要甲、乙不相鄰,且丙、丁也不相鄰,則有 $\blank{5em}$ 種排法。
  \end{enumerate}  
\end{enumerate}

\newpage
\section{混合題}
\begin{enumerate}[label=\arabic*.,align=left,leftmargin=*,labelsep=.3em,start=3]
  \item 考慮正整數 $N = 10800$。
  \begin{enumerate}[label=(\arabic*),align=left,leftmargin=*,labelsep=.3em]
    \item 已知 $N = 2^4 \times 3^3 \times 5^2$。設 $a$、$b$、$c$ 皆為非負整數,則下列何者是「$2^a \times 3^b \times 5^c$ 是 $N$ 的因數」的充分必要條件?(單選)
    \begin{enumerate}[label=(\Alph*),align=left,leftmargin=*,labelsep=.3em]
      \item $0 \leq a + b + c \leq 8$
      \item $0 \leq a + b + c \leq 9$
      \item $0 \leq a \leq 3$ 且 $0 \leq b \leq 2$ 且 $0 \leq c \leq 1$
      \item $0 \leq a \leq 4$ 且 $0 \leq b \leq 3$ 且 $0 \leq c \leq 2$
      \item $1 \leq a \leq 4$ 且 $1 \leq b \leq 3$ 且 $1 \leq c \leq 2$
    \end{enumerate}
    \item 共有 $\blank{5em}$ 個正整數是 $N$ 的因數。
  \end{enumerate}
\end{enumerate}

\newpage
\label{answer}
{\bfseries\large 答案 \par}
%\setcounter{section}{0}
%\section{填充題}
\begin{enumerate}[label=\arabic*.,left=0pt]
  \item
  \begin{enumerate}[label=(\arabic*),align=left,leftmargin=*,labelsep=.3em]
    \item 16
    \item 14
    \item 4
  \end{enumerate}
  \item
  \begin{enumerate}[label=(\arabic*),align=left,leftmargin=*,labelsep=.3em]
    \item 720
    \item 96
    \item 336
  \end{enumerate}
  \item
  \begin{enumerate}[label=(\arabic*),align=left,leftmargin=*,labelsep=.3em]
    \item (D)
    \item 60
  \end{enumerate}
\end{enumerate}
\end{document}
